\documentclass[10pt,a4paper,twocolumn]{article}\usepackage[]{graphicx}\usepackage[]{color}
%% maxwidth is the original width if it is less than linewidth
%% otherwise use linewidth (to make sure the graphics do not exceed the margin)
\makeatletter
\def\maxwidth{ %
  \ifdim\Gin@nat@width>\linewidth
    \linewidth
  \else
    \Gin@nat@width
  \fi
}
\makeatother

\definecolor{fgcolor}{rgb}{0.345, 0.345, 0.345}
\newcommand{\hlnum}[1]{\textcolor[rgb]{0.686,0.059,0.569}{#1}}%
\newcommand{\hlstr}[1]{\textcolor[rgb]{0.192,0.494,0.8}{#1}}%
\newcommand{\hlcom}[1]{\textcolor[rgb]{0.678,0.584,0.686}{\textit{#1}}}%
\newcommand{\hlopt}[1]{\textcolor[rgb]{0,0,0}{#1}}%
\newcommand{\hlstd}[1]{\textcolor[rgb]{0.345,0.345,0.345}{#1}}%
\newcommand{\hlkwa}[1]{\textcolor[rgb]{0.161,0.373,0.58}{\textbf{#1}}}%
\newcommand{\hlkwb}[1]{\textcolor[rgb]{0.69,0.353,0.396}{#1}}%
\newcommand{\hlkwc}[1]{\textcolor[rgb]{0.333,0.667,0.333}{#1}}%
\newcommand{\hlkwd}[1]{\textcolor[rgb]{0.737,0.353,0.396}{\textbf{#1}}}%

\usepackage{framed}
\makeatletter
\newenvironment{kframe}{%
 \def\at@end@of@kframe{}%
 \ifinner\ifhmode%
  \def\at@end@of@kframe{\end{minipage}}%
  \begin{minipage}{\columnwidth}%
 \fi\fi%
 \def\FrameCommand##1{\hskip\@totalleftmargin \hskip-\fboxsep
 \colorbox{shadecolor}{##1}\hskip-\fboxsep
     % There is no \\@totalrightmargin, so:
     \hskip-\linewidth \hskip-\@totalleftmargin \hskip\columnwidth}%
 \MakeFramed {\advance\hsize-\width
   \@totalleftmargin\z@ \linewidth\hsize
   \@setminipage}}%
 {\par\unskip\endMakeFramed%
 \at@end@of@kframe}
\makeatother

\definecolor{shadecolor}{rgb}{.97, .97, .97}
\definecolor{messagecolor}{rgb}{0, 0, 0}
\definecolor{warningcolor}{rgb}{1, 0, 1}
\definecolor{errorcolor}{rgb}{1, 0, 0}
\newenvironment{knitrout}{}{} % an empty environment to be redefined in TeX

\usepackage{alltt}
\usepackage{f1000_styles}
\usepackage{hyperref}

\newcolumntype{L}{>{\raggedright\arraybackslash}m{5cm}} % creates now column type to wrap text
\IfFileExists{upquote.sty}{\usepackage{upquote}}{}


\begin{document}




\title{spocc: taxonomic search and retrieval in R}
\author[1]{Edmund Hart}
\author[2]{Scott Chamberlain}
\affil[1]{NEON}
\affil[2]{Portland}

\maketitle
\thispagestyle{fancy}

\begin{abstract}
XXXXXX.
\end{abstract}
\clearpage

\section*{Introduction}
write stuff....

\section*{spocc is cool}
XXXXXX

\section*{Data sources and package details}
XXXXXX

\section*{Use cases}

\subsection*{First, install spocc}

First, one must install and load spocc into the R session.

\begin{knitrout}\scriptsize
\definecolor{shadecolor}{rgb}{0.969, 0.969, 0.969}\color{fgcolor}\begin{kframe}
\begin{alltt}
\hlkwd{install.packages}\hlstd{(}\hlstr{"spocc"}\hlstd{)}
\hlkwd{library}\hlstd{(spocc)}
\end{alltt}
\end{kframe}
\end{knitrout}





Advanced users can also download and install the latest development copy from GitHub \url{https://github.com/ropensci/spocc}.

\subsection*{XXXXX}

XX

\begin{knitrout}\scriptsize
\definecolor{shadecolor}{rgb}{0.969, 0.969, 0.969}\color{fgcolor}\begin{kframe}
\begin{alltt}
\hlkwd{print}\hlstd{(}\hlstr{"hello world"}\hlstd{)}
\end{alltt}
\begin{verbatim}
# [1] "hello world"
\end{verbatim}
\end{kframe}
\end{knitrout}


\subsection*{XXXX}

XXXX

\section*{Conclusions}
XXXXX

\subsection*{Author contributions}
XXXXX

\subsection*{Competing interests}
No competing interests were disclosed.

\subsection*{Grant information}
The author(s) declared that no grants were involved in supporting this work.

\subsection*{Acknowledgements}
The spocc package is part of the rOpenSci project \url{http://ropensci.org/}. We thank all API maintainers for their work making their databases open to the public.

\nocite{*}
{\small\bibliographystyle{unsrt}
\bibliography{refs_nourls}}

\end{document}
